\documentclass[12pt,a4paper]{article}
\usepackage[utf8]{inputenc}
\usepackage{amsfonts}
\usepackage{amsmath}
\usepackage{mathtools}
\usepackage{amsthm}
\usepackage{amssymb}
% Esto es una definición para hacer que las "|" en las matrices se vean mejores 

\makeatletter
\renewcommand*\env@matrix[1][*\c@MaxMatrixCols c]{%
  \hskip -\arraycolsep
  \let\@ifnextchar\new@ifnextchar
  \array{#1}}
\makeatother

%Estos son los comandos para escribir un teorema o una definición, respectivamente 

\newtheorem{teo}{Teorema}
\newtheorem*{defi}{Definición}
\newtheorem*{prop}{Proposición}

%Autor, título, fecha (usar \maketitle donde lo quieras)
\title{Series de Potencias}
\author{Jhonny Lanzuisi}
\date{23 de Febrero de 2018}

\begin{document}
\maketitle
	
Continuamos con el desarrollo en series de Taylor de funciones notables.

Para \underline{\(f(x) = cos(x)\)} tenemos que:
\begin{equation}
cos (x) = 1 - \frac{x^2}{2!} + \frac{x^4}{4!} + \dots + (-1)^n \frac{x^{2n}}{2n!} + \int_{0}^{x} \frac{cos^{2n+1}(t)}{2n!} (x-t)^{2n} dt
\end{equation} 	
Así el 
\begin{equation*}
|R_n(cos(x),0)| \leq \frac{|x|^{2n}}{(2n+1)!}
\end{equation*}

Para \underline{\(f(x) = e^x\)}tenemos que:

\begin{equation}
e^x = 1+x+\frac{x^2}{2!} + \dots + \frac{x^n}{n!} + \int_{0}^{x} \frac{e^t}{n!} (x-t) dt
\end{equation}
Así el 
\begin{equation*}
|R_n(e^x,0)| \leq \frac{e^x|x|^{n+1}}{(n+1)!}
\end{equation*}

Para \underline{\(f(x) = arctg(x)\)} tenemos:
\begin{align*}
arctg(x) = \int_{0}^{x} \frac{1}{1+t^2} dt &= \int_{0}^{x} \sum_{n=0}^{\infty} (-t^e)^n dt =\sum_{n=0}^{\infty} \int_{0}^{x} (1-t^2)^n =\\
\sum_{n=0}^{\infty} \frac{(-1)^n t^{2n+1}}{2n+1} \Big|_0^x &= \sum_{n=1}^{\infty} (-1)^n \frac{x^{2n+1}}{2n+1}
\end{align*}
en resumen 
\begin{equation}
arctg(x) = \sum_{n=0}^{\infty} \frac{(-1)^n}{2n+1} x^{2n+1}
\end{equation}
si \( |x| < 1 \). Su radio de convergencia es:
\begin{equation*}
R = \frac{1}{\lim\limits_{x \rightarrow \infty} \sqrt[n]{\frac{(-1)^n}{2n+1}}} = \lim\limits_{x \rightarrow \infty} \sqrt[n]{2n+1} = 1
\end{equation*}

A continuación esta el teorema sobre la unicidad de la representación en series de potencias
\begin{teo}
	Si dos series de potencias 
	\begin{equation*}
	\sum_{n=0}^{\infty} a_n (x-a)^n \quad \text{y} \quad \sum_{n=0}^{\infty} b_n (x-a)^n
	\end{equation*}
	tienen la misma función suma f en un cierto entorno del punto a, \textbf{entonces las dos series son de iguales términos.}
\end{teo}
\end{document}
